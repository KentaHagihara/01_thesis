\documentclass[10pt]{jarticle}
\usepackage[dvipdfmx]{graphicx}
\usepackage[dvipdfmx]{color}
\usepackage{here}
\usepackage[top=30truemm, bottom=20truemm, left=30truemm, right=30truemm]{geometry}



\title{原子核の変形と中性子ドリップラインに対するクーロン相互作用の効果}
\author{萩原健太}
\date{2023年1月21日}
\begin{document}

\begin{center}
  {\Large
    原子核の変形と中性子ドリップラインに対する\\
    クーロン相互作用の効果
  }
\end{center}

\begin{flushright}
  \vskip\baselineskip201910867 萩原健太\\
  指導教員:中務孝 日野原伸生
\end{flushright}

\section*{概要}
エネルギー汎関数理論の発展により、多数の陽子と中性子が集まった量子多体系である原子核をより正確に記述することが出来るようになってきた。
この理論を用いることで、比較的小さな質量数の原子核から超重核と呼ばれる質量数の大きな原子核に対して、系統的な数値計算の実行が可能である。
Ebata and Nakatsukasa,Phys.Scr.92,064005 (2017)の論文内で陽子数$Z=6-92$の同位体のうち$N=Z$と$N=2Z$の範囲の原子核について、3D Skyrme Hartree-Fock  plus BCS modelに基づいた計算結果が報告されている。
我々のグループでは計算範囲をさらに広げ、陽子数$Z=2-120$の同位体のうち陽子、中性子ともにchemical potentialが負であるような原子核について計算を実行した。
計算には、軸対象変形について解くことが出来るHFBTHO programmeを利用することで、クーロン相互作用が偶偶核の変形度および中性子ドリップラインに与える影響について評価した。\\
 計算の結果からクーロン相互作用により、原子核全体で変形度が大きくなる傾向にあること、そして興味深いことに、中性子ドリップラインが拡大することが判明した。
斥力の効果からエネルギー的には不安定になることが直感的に予想されるが、結果的に束縛状態にある原子核を増やすような計算結果を得た。
また中性子ドリップラインが拡大している領域において、変形度が増加している原子核だけでなく球形核の状態で存在するものも確認され、中性子側の存在限界と変形度は必ずしも直接的な相関があるとは言えないことも興味深い。
これらのプロセスについて考察を行うために、原子核の半径やエネルギー、変形度など様々な物理量を評価した。\\
\end{document}