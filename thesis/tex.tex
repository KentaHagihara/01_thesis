% yitp00a.tex : 00/1/12,13
% For proceedigs of a work shop held at Yukawa inst. for theor. phys. (YITP)
% in Nov 10-12, 1999
% My talk was on Nov 12 from 13:20.
% Soryushiron Kenkyu Vol?? pp. B81 - B85.
% 「不安定核の構造と反応」1999年11月10~12日 於 京都大学基礎物理学研究所
%
\documentstyle[12pt]{article}
\topmargin=-14mm
\oddsidemargin=-5mm   % side margin for odd-page-number page
\evensidemargin=-5mm  %                 even
\textheight=247mm % for Camera-ready manuscript for Soryuushiron-Kenkyu
\textwidth=170mm  %
\newcommand{\mysection}[1]{\vspace{\baselineskip} \noindent \underline{#1} \\}
\newcommand{\vecrp}[0]{{\vec{r}\,}'}
\begin{document}
\baselineskip=0.6cm
% \pagestyle{empty}

\begin{center}
{\Large
座標表示HFB法による中性子過剰核の対相関
}
\end{center}

% \vspace{\baselineskip}

\begin{flushright}
田嶋 直樹 (福井大学・工学部)
\end{flushright}

% \vspace{\baselineskip}

\mysection{1. 序}

中性子滴下線近傍の原子核では、中性子の対相関により、基底状態波動関
数への連続状態成分の混入が増え、その効果は核の諸性質に大きな影響
を与えるだろうと予想されている。

これらの核を対相関を含む平均場法である Hartree-Fock-Bogoliubov (HFB)法
で扱うには、これらの核に特有な広くひろがった密度分布に配慮しなければな
らない。このため、調和振動子などの基底で張られた配位空間での対角化によ
る解法は、簡便なようだが、実は非能率的であり、正方メッシュ点上の値で波
動関数を表現するなどの座標空間での解法\cite{BFH85}が適していると思われ
る。

しかし、波動関数を座標空間で表現すると、連続スペクトル中の一体状態は核
のまわりに局在せず規格化箱全体に広がって膨大な状態密度を持つことになる。
このことは、局在した状態のみを対象としてきたこれまでの平均場の解法
には大きな困難となる\cite{THF96,Taj98a}。

% \footnote{
% この効果を HF+BCS 近似で扱おうとすると原子核の周囲を中性子のガスがとり
% まく現実に則さない解が得られる。核子の密度分布が原子核のまわりに局在し
% た解を得るにはHFB法を近似なしに解くことが肝要である。
% }

私はHFB正準基底によるHFBの解法がこの困難を解消する切札であると考え、こ
れまでにも研究会等でその詳細を報告してきたが
\cite{Taj98a,Taj98ws1,Taj00a}、本講演では、私の研究の最近の進展から特
にカットオフに関する件と、この手法で実際の中性子過剰核の性質が問題なく
計算されることに絞って報告を行なった。本報告書に載録するのは、その講演
内容のうち、カットオフの必要性および(他所にはあまり書いていない)相互作
用への状態依存因子の導入によるカットオフの導入方法の得失についての説明
である。実際に計算が問題なく行なえることの検証結果は執筆中の論文
\cite{Taj00b}に載録することにしたい。

% -----------------

なお連続状態の対相関への寄与は核子滴下線近傍という限定された核の記述に
のみ必要なのではない。
存在可能とされる$10^4$もの核種の約半数では、中性子のフェルミ準位から連続
スペクトルの始まるエネルギー準位(=0)までの間隔が主殻の間隔
($\hbar \omega = 41 A^{-1/3}$MeV)の半分以下しかない。
殻補正法では、殻効果を定量的に取り入れるには一主殻以上の配位空間が必要
であると言われる。対相関の計算は一主殻より小さい空間でされることが多い
が、それでも主殻の半分は必要というのが経験の教えるところである。したがっ
て、全原子核の半数について、対相関を精度良く扱おうとすると連続状態を
取り入れる必要があることになる。

% ------------------------

\mysection{2. HFBの正準基底表現による解法}

座標空間でのHFBの解法は文献\cite{DFT84}にさかのぼる。
そこでは準粒子状態の座標空間での波動関数を用いてHFB解を
表した:
\vspace*{-1mm}
\begin{eqnarray}
\label{eq:psi_qp}
|\psi \rangle & = & \prod_{i=1}^{\# {\rm basis}} b_i | 0 \rangle, \\
\label{eq:qp}
b_i & = & \sum_{s} \int d \vec{r} \left\{
\phi_i^{\ast} (\vec{r},s) \; a_{s}(\vec{r}) +
\psi_i (\vec{r},s) \; a^{\dagger}_{s}(\vec{r}) \right\}
\;\;\; \mbox{ : 準粒子状態}.
\end{eqnarray}
\vspace*{-5mm}

\noindent
この方法はSkyrme型の相互作用\footnote{
%
相互作用の到達距離が零であるため平均場が局所的になることが座標表示解を
求めるのに決定的な利点となる。もし到達距離が有限ならば交換項が非局所的
なポテンシャルを作り、その結果、考慮する一粒子軌道の対の数だけの交換ポテ
ンシャルが必要になる。これはごく軽い核以外では実質的に実行不可能な
計算量を要求する。\cite{Mat98}
%
}
を用いて球形核への適用されたが、
その変形核への適用は準粒子状態数の膨大さのために難しい。一方、
Bloch-Messiah定理により上記の解はBCS変分関数の形にも表すことができる:
\vspace*{-1mm}
\begin{eqnarray}
\label{eq:psi_cb}
|\psi \rangle & = & \prod_{i=1}^{K}
\left( u_i + v_i \; a^{\dagger}_{i} \; a^{\dagger}_{\bar{\imath}} \right)
| 0 \rangle, \;\;\; K=  (\# {\rm basis})/2,\\
\label{eq:cb}
a^{\dagger}_i & = & \sum_{s} \int d \vec{r} \; \psi_{i} (\vec{r},s) \;
a^{\dagger}_{s}(\vec{r})
\;\;\; \mbox{ : HFB 正準基底}.
\end{eqnarray}

\vspace*{-1mm}

\noindent
偶々核の基底状態では時間反転対称性のため
$\psi_{\bar{\imath}}$は$\psi_{i}$を時間反転操作したものになる。
この表現による解法は文献\cite{RBR97}で導入され球形核に適用された。
私はこれを変形核に適用した\cite{Taj98a,Taj98ws1,Taj00a}。私の手法の
特徴を下記に箇条書きする。

\vspace*{2mm}

\noindent
i) 「変形」と「連続状態の対相関」とを共に扱える新しい手法である。

\vspace*{2mm} 

\noindent
ii) 調和振動子などの基底関数による展開によらず3次元正方メッシュによっ
て波動関数を表すので変形やハロー発生に合わせた基底の最適化という微妙な
調整操作が不要である。

\vspace*{2mm} 

\noindent
iii) 必要な一体波動関数の数は、準粒子による表現では規格化箱の体積に
比例するのに対し、正準基底による解の表現では核の体積に比例するので計算
量が数十分の一に軽減できる。

\vspace*{2mm} 

\noindent
iv) 波動関数のスケーリングによる最急降下法の加速と、それに適応した
直交性拘束条件のためのラグランジュ乗数の汎関数形を見出した。

\vspace*{2mm} 

\noindent
v) 到達距離が零の力を用いる場合、正準基底数の制限はカットオフの完全な代用
にならないことを見出した。高エネルギーの正準基底状態の波動関数が点状に
収縮して対相関エネルギーの発散が起きうるのである。最善の回避策として対
相関密度依存相互作用の導入を提案した。

\vspace*{2mm} 

上記の v) について次節以降で説明する。

\mysection{3. HFB正準基底数の制限はカットオフを代用するか}

正準基底によるHFB方程式の解法とは、正準基底(\ref{eq:cb})の波動関数
$\{ \psi_i \}$を変分させることで、BCS型状態(\ref{eq:psi_cb})をHFB解へ
と収束させようとするアプローチである。

(\ref{eq:psi_cb})式の乗積$\Pi$で掛け合わせる項の数$K$は、原
理的には一粒子波動関数 $\psi_i$を定義する空間の次元の半分となるが、実
際の計算では、それより遥かに小さい数(例えば粒子数と同数)に設定できるはずであ
り、そう設定できることがこの解法の最大の利点である\cite{Taj98a}。

さて、到達距離が零である相互作用を用いるときには対相関にカットオフを導入す
る必要がある\cite{TOT94}。しかし、もし、正準基底の数を有限個に制限すれ
ば、カットオフがなくても対相関は発散せず物理的に意味のある基底状態が得
られるのではないか、という期待も湧くであろう。
しかし(\ref{eq:psi_cb})式の変分関数でのハミルトニアンの期待値を最急
降下法で極小化するコンピュータプログラムを実際に作製し多種多様な場合に
適用した結果対相関ギャップが異常に大きくなるケースが散見されたのである。
即ち長ステップにわたって最急降下法\footnote{
%
最急降下の方向は、変分空間が同じでも、それを定義するパラメータの設定
のしかたに依存して変わるが、設定方法を変えても、同様の現象が
起きる。
%
}
による発展を続けると、収束したように見えたエネルギーなど
の量に急にジャンプがおきることがある。その原因を調べてみると、高いエ
ネルギー期待値を持ったHFB正準軌道の波動関数が、あるとき急激に
点状\footnote{
%
メッシュサイズが有限なので実際には点にはならず、メッシュサイズで収
縮は止まる。なお、プログラムでは
フーリエ基底に基づく微積分の扱いをしているためこのよ
うな波動関数も正当に扱われている。
なお、多項式近似による微分値を用いると、高波数状態のエネルギーが過小評
価されるため点状収縮は助長されるはずである。
%
}
に縮小する現象のためだと分った。

このような収縮が起きる理由は単純である。
ハミルトニアン密度は、運動エネルギー密度 $\tau$、密度 $\rho$、
対密度 $\tilde{\rho}$などを用いて書けるが、
$i$番目の正準基底軌道の占拠振幅 $v_i$、$u_i$に対するそれぞれの
密度の依存性は、
%
\begin{equation}
\tau, \rho \propto v_i^2, 
\;\;\;
\tilde{\rho} \propto u_i v_i,
\end{equation}
%
である。高いエネルギーの状態については、$v_i \ll 1, u_i \doteq 1$
であるから $u_i v_i \gg v_i^2$ が成り立つ。点状収縮による運動
エネルギーの損失は $v_i^2$に比例し、$u_i v_i$に比例する対相関エネル
ギーでの利得より遥かに小さいので、点状収縮が起き得るのである。

一部ではあっても一粒子波動関数が点状に収縮しているような解は物理的に無
意味である。ただし、ほとんどの場合、点状収縮の前に、ほぼ安定な解が得ら
れている\footnote{
%
しかし、最急降下法の取り方によって、安定性に大きな変動が見られる。
また、準安定状態が得られない場合もあるかもしれない。
状況によっては、いくつかの正準基底波動関数が点状に収縮した物理的に無意
味な解と有意味な状態の間にエネルギー上の障壁があるかもしれない。その場
合は、有意味な空間のみを考えればよいので、カットオフは不要であろう。し
かし、計算結果からわかったように、障壁がない場合が多い。あるいは、全て
の場合に障壁がなく、長い 最急降下法による発展をさせれば、必ず点状収縮
への崩壊が起きている可能性もある。
%
}
ので、そのような準安定状態が得られた場合には、それで解を代用すればよい
という考え方もあるであろう。しかし、この平均場模型の解法が広く利用され
るような信頼性のあるものになるためには、解を求めることに頻繁に失敗する
ようなものでは不十分で、確実に物理的に意味のある解が求まるようにしてお
かねばならないと思う。そこで、カットオフをうまく導入して、多くの場合に
現れる準安定状態に近い真の安定な解を作ることを目標に研究を進めることに
した。

そこで、Skyrme力のように到達距離が零の有効相互作用にどのようにしてカッ
トオフを導入するかについて下記の3通りの方法を考えた。

\vspace*{2mm}

\noindent
1) 相互作用に波数依存性を状態依存因子で近似して導入する。

\vspace*{2mm}

\noindent
2) 相互作用に対密度依存性を導入する。\\
有効相互作用の密度依存性の考え方を対密度
依存性にまで拡張しようとするものである。対相互作用のエネルギー密度を下
記のように拡張する。
%
\begin{equation}
\tilde{\cal H}_{\rm c} = \frac{1}{8} v_{\rm p} 
\left( 1-\frac{\rho\left(\vec{r}\right)}{\rho_{\rm c}}
-\left(\frac{\tilde{\rho}\left(\vec{r}\right)}{\tilde{\rho}_{\rm c}}\right)^2
\right)
\left\{\tilde{\rho} (\vec{r})\right\}^2
\end{equation}
%
$\rho_{\rm c}$は相互作用がquenchされてちょうど零になる核子密度を表すパ
ラメータであり、$\tilde{\rho}_{\rm c}$は同様のことを対密度にも拡張した
パラメータである。
通常の密度依存項では、原子核中の密度とパラメータ $\rho_{\rm c}$は
同程度の大きさになるが、対密度依存項では、
パラメータ$\tilde{\rho}_{\rm c}$は
原子核中の対密度 $\tilde{\rho}$より一桁大きい値で十分に点状収縮の
阻止\footnote{
%
この項の作用で点状収縮解はエネルギーがむしろ高くなるので、異常な解への転落を
止めるというより異常な解をなくしてしまうと考えたほうがよい。
%
}
の役割を果たすことができることがわかっている。

\vspace*{2mm}

\noindent
3) 運動量依存性によりカットオフをより自然なものに近づける。\\
Skyrme相互作用の $\frac{1}{2} t_1 \left( \vec{k}^2 \delta + \delta
\vec{k}^2 \right)$ 項は、高い運動量の状態間の遷移に対して斥力的な寄与
をする。この項と真のカットオフとの併用により相互作用到達距離による自然
なカットオフに類似したものを実現できる可能性が期待される\cite{TOT94}。

\vspace*{2mm}

案 1) で点状収縮問題は解決できたが定式化に不満な点があった。これは他所で
詳述していないので次節で説明することにする。。
その後 案 2)でほぼ完全に満足の行くカットオフ法を得ることができた。これに関して
は文献\cite{Taj00a}に詳述したのでそれを参照されたい。
さらにいずれ案 3) も検討する予定である。

\mysection{4. 相互作用到達距離の有限性を状態依存因子で近似する}

Seniority力の場合、カットオフは下記のように導入すると解法の変更が
ほとんど不要なので便利である。
%
\begin{equation}
V_{\rm pair} = -G 
\left(\sum_{i >0} a^{\dagger}_{i} a^{\dagger}_{\bar{\imath}} \right)
\left(\sum_{j >0} a_{j} a_{\bar{\jmath}} \right)
\;\;\; \Rightarrow \;\;\;
-G 
\left(\sum_{i >0} f_i a^{\dagger}_{i} a^{\dagger}_{\bar{\imath}} \right)
\left(\sum_{j >0} f_j a_{j} a_{\bar{\jmath}} \right)
\end{equation}
%

上式で$f_i$ は $i$番目の正準軌道の対相関相互作用の減衰因子であり、(対
相関を含まないHartree-Fock場だけの)平均場近似の一粒子ハミルトニアンのエネ
ルギー準位$\epsilon_i$の関数として高エネルギーで零になるようにとるこ
とが多い\footnote{
%
Pairing channelにおける相互作用が状態の波動関数に無関係に決まるとき、
たとえば、seniority力のようであるときにHartree-Fock-Boguliubov方程式は
HF+BCS方程式に簡略化できる。しかし、$f_i$ が $\epsilon_i = \langle i |
h | i \rangle$の関数であるとすれば、相互作用の行列要素は状態の波動関数
$\left\vert i \right\rangle$ に依存するので、カットオフ因子付の相互作
用の場合にはHFBのHF+BCSへの置き換えは近似的なものに過ぎない。
%
}。
%
接触力の場合に、これと同様の方法でカットオフを導入することは
対密度の定義にカットオフ因子を導入することに他ならない\footnote{
%
このような修正後のエネルギー密度汎関数をあたえる相互作用は非常に人為的
なものになるであろう。しかし、エネルギー密度汎関数は相互作用から決定さ
れるものであるが、計算はもっぱらエネルギー密度汎関数を経由して行われる
ため、解を求めるにあたっては元にある相互作用がどのようなものであるかを
思い悩む必要はない。
%
}。
%
\begin{equation}
\tilde{\rho}(\vec{r}) = \sum_{i>0} u_i v_i |\psi_i (\vec{r})|^2
\;\;\; \Rightarrow \;\;\;
\tilde{\rho}_{\rm c}(\vec{r}) = \sum_{i>0} f_i u_i v_i |\psi_i (\vec{r})|^2.
\end{equation}
%

波動関数の点状収縮を回避するという目的に照らすと、
$f_i$が波動関数に依存することからくる変分を無視せずに
正確に取り入れることが肝要であると思われる。
そのために、エネルギー$\epsilon_i$によって$f_i$を決めたのでは
変分の停留条件を表す方程式に非常に複雑な項が出現するので、
むしろ波数ベクトル大きさの期待値 $k_i^2 =
- \int \psi_{i}^{\ast} (\vec{r}) \triangle_{\vec{r}} \psi_i (\vec{r}) 
d \vec{r}$への依存性を採用するのが賢明である。即ち、
%
\begin{equation}
f_i = f (k_i^2) = \exp \left( -\frac{\mu^2}{4}k_i^2 \right).
\end{equation}
%
上式で$\mu$は、対相互作用を定量的に正しく記述すると言われ
る Gogny D1力\cite{DG80}の長距離引力の動径依存性と同じ、1.2fmにとった。
点状収縮した波動関数はメッシュ点間隔程度の波数を持つので、非常に
高い波数がカットされることで、解の崩壊が抑止できると期待できる。実
際、この形でカットオフを導入し変分計算を正確な最急降下路を用いて行なった
ところ、解は点状に崩壊することなく、物理的に尤もな解が得られることが示
された。

しかしながら、この形でのカットオフ導入の短所として、状態依存した
ハミルトニアン密度を設定したことに起因して、すべての一準粒子状態をその固
有状態として与える一準粒子ハミルトニアンが存在しないことが挙げられる。
一準粒子ハミルトニアンとは、運動エネルギーとHartree-Fockポテンシャ
ルからなる$h$と、対相関ポテンシャルからなる $\tilde{h}$を $2 \times 2$の行列
にまとめたものであり、下記のような形で与えられる\footnote{
%
時間反転不変性を仮定して、時間反転対のうち片方のみを考えるように
簡略化してあるのでHermiteな行列になっている。
%
}。
%
\begin{equation}
 h_{qp} = 
\left(
\begin{array}{cc}
h & \tilde{h} \\
\tilde{h}& h
\end{array}
\right)
\end{equation}
%
HFB解を求めるにあたっては、最急降下法による変分に加えて、一準粒子ハミ
ルトニアンの対角化を併用することが解への収束を早めたり、局所解へのトラッ
プを防止するのに有効であることが経験的に分かっているのだが、その対角化
が正確にはできなくなるのである\footnote{
%
平均的な一準粒子ハミルトニアンを定義すれば、低精度な解の段階では役に立つが、
高精度な解になると、対角化によってむしろ解から遠ざかる結果になる。
%
}。

なお、このような準粒子ハミルトニアンの不定性は、
実は文献\cite{DFT84}においても既に、準粒子の真空としてHFB解を表したとき
準粒子をエネルギーで選別することで
内包されていたのである\cite{Rin99}。

上記の不満から、私は対密度依存性によるカットオフの導入
を思いつき、それがむしろ好ましいと考えるように到ったのである。

\begin{thebibliography}{99}
{\small
   \setlength{\labelwidth}{8mm}
   \setlength{\labelsep}{0.5mm}
   \setlength{\leftmargin}{8.5mm}
   \setlength{\rightmargin}{0mm}
   \setlength{\listparindent}{0mm}
   \setlength{\parsep}{0mm}
   \setlength{\itemsep}{0mm}
\baselineskip=0.5cm
\bibitem{BFH85}  % Self consistent study of triaxial deformations
         P.~Bonche, H.~Flocard, P.-H.~Heenen, S.J.~Krieger, and M.S.~Weiss,
         Nucl.\ Phys.\ {\bf A443} (1985) 39.
\bibitem{THF96} % 3D solution to HFB for drip line nuclei
         J.~Terasaki, Heenen, Flocard and Bonche,
         Nucl.\ Phys.\ {\bf A600} (1996) 371.
\bibitem{Taj98a} % Hartree-Fock-Bogoliubov for deformed neutron-rich nuclei
        N.~Tajima,
        proc. XVII RCNP Int. Sym. on Innovative Computational Methods
        in Nuclear Many-Body Problems, Osaka, 1997,
        edited by H. Horiuchi, M. Kamimura, H. Toki, F. Fujiwara, M. Matsuo,
        and Y. Sakuragi,
        (1998) World Scientific (Singapore) , pp. 343-351
\bibitem{Taj98ws1} %
         N. Tajima, proc. of a workshop at YITP in January 19-21, 1998, 
         素粒子論研究97巻2号(1998年5月) pp.B9-B12
\bibitem{Taj00a} % Pairing correlation involving the continuum states
        N.~Tajima,
        proc.\ Int.\ Sym. on Models and Theories of the Nuclear Mass,
        Wako, Japan, 1999,
        RIKEN Review No.26 (2000.1) pp. 87-94.
\bibitem{Taj00b} %
        N.~Tajima,
        in preparation.
\bibitem{DFT84} % HFB description of neutron drip line, SkP force
         J.~Dobaczewski, Flocard and Treiner,
         Nucl.\ Phys.\ {\bf A422} (1984) 103.
\bibitem{Mat98} %
        T.~Matsuse,
        proc. XVII RCNP Int. Sym. on Innovative Computational Methods
        in Nuclear Many-Body Problems, Osaka, 1997,
        edited by H. Horiuchi, M. Kamimura, H. Toki, F. Fujiwara, M. Matsuo,
        and Y. Sakuragi,
        (1998) World Scientific (Singapore) , pp. 369-371
\bibitem{RBR97} % An HFB scheme in natural orbitals
         P.-G.~Reinhard, Bender, Rutz and Maruhn,
         Z.\ Phys.\ {\bf A358} (1997) 277.
\bibitem{TOT94} % nmbcs
         S.~Takahara, N.~Onishi, and N.~Tajima,
         Phys.\ Lett.\ {\bf B331} (1994) 261.
\bibitem{DG80} % Gogny D1 force
         J.~Decharg\'{e} and D.~Gogny,
         Phys. Rev. {\bf C21} (1980) 1568.
\bibitem{Rin99} %
        P.~Ring,
        comment in discussion time after N.~Tajima's talk in
        Int.\ Sym. on Models and Theories of the Nuclear Mass,
        RIKEN, Wako, Saitama, Japan, July, 1999.
}
\end{thebibliography}

\end{document}
