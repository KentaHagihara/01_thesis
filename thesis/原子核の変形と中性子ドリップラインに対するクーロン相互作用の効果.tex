\documentstyle[12pt]{article}
\topmargin=-14mm
\oddsidemargin=-5mm   % side margin for odd-page-number page
\evensidemargin=-5mm  %                 even
\textheight=247mm % for Camera-ready manuscript for Soryuushiron-Kenkyu
\textwidth=170mm  %
\newcommand{\mysection}[1]{\vspace{\baselineskip} \noindent \underline{#1} \\}
\newcommand{\vecrp}[0]{{\vec{r}\,}'}
\begin{document}
\baselineskip=0.6cm
% \pagestyle{empty}

\begin{center}
{\Large
原子核の変形と中性子ドリップラインに対するクーロン相互作用の効果
}
\end{center}

% \vspace{\baselineskip}

\begin{flushright}
Kenta HAGIHARA (University of Tsukuba)
\end{flushright}

% \vspace{\baselineskip}

\mysection{1. introduction}
エネルギー密度汎関数法を用いることで、核図表上のあらゆる原子核の性質を系統的に解析することができる。
本論文ではHFBTHOプログラム(Skyrme-Hartree-Fock-Bogoliubov方程式の変形調和振動子基底による軸方向変形解(II))を用いて、偶-偶核とドリップラインの変形に対するクーロン相互作用の影響を報告する。
超重核を含む原子核について、陽子および中性子のドリップラインまでの計算を行った結果、クーロン相互作用は質量数の大きな領域で原子核の変形を増大させ、ドリップラインを中性子側に伸ばすことがわかった。
また、クーロン相互作用が中性子ドリップライン近傍の原子核に付加的な束縛を与えていることも興味深い。
これらの効果の微視的なメカニズムを理解するために、変形や半径などの制約を加えた計算結果を報告する。

\mysection{2. purpose of study}
本研究の目的はクーロン相互作用が陽子・中性子ともに偶数の原子核のドリップライン、および変形度に与える影響を明らかにすることである。
またドリップラインや変形度に加えて原子核の半径やペアリングギャップを調べることで、クーロン相互作用が原子核に与える微視的な影響を探求する。
そしてクーロン相互作用が中性子ドリップライン付近の原子核に与える付加的な束縛についても考察を行う。

\mysection{3. methods}
軸対象の変形を扱うことが出来るSkyrme型原子核密度汎関数法計算コードであるHFBTHOプログラムを用いて、陽子数2から120の偶偶核について計算を実行する。

\mysection{4. results}

\mysection{5. discussion}

\mysection{6. conclusion}

\mysection{7. references}

\mysection{8. appendices}

\end{document}
